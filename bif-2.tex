% Created 2020-03-24 Tue 14:22
% Intended LaTeX compiler: pdflatex
\documentclass[11pt]{article}
\usepackage[utf8x]{inputenc}
\usepackage[T1]{fontenc}
\usepackage{graphicx}
\usepackage{grffile}
\usepackage{longtable}
\usepackage{wrapfig}
\usepackage{rotating}
\usepackage[normalem]{ulem}
\usepackage{amsmath}
\usepackage{textcomp}
\usepackage{amssymb}
\usepackage{capt-of}
\usepackage{hyperref}
\usepackage{minted}
\author{Jakub Zárybnický (xzaryb00@stud.fit.vutbr.cz)}
\date{\today}
\title{Zarovnání sekvencí}
\hypersetup{
 pdfauthor={Jakub Zárybnický (xzaryb00@stud.fit.vutbr.cz)},
 pdftitle={Zarovnání sekvencí},
 pdfkeywords={},
 pdfsubject={},
 pdfcreator={Emacs 26.3 (Org mode 9.1.9)}, 
 pdflang={Czech}}
\begin{document}

\maketitle
\tableofcontents


\section{Dotlet}
\label{sec:org7bdc741}
Seznamte se s programem \href{https://dotlet.vital-it.ch/}{Dotlet}. Zadejte programu různé (např. náhodné) vstupní
sekvence (\href{http://www.bioinformatics.org/sms2/random\_dna.html}{nukleotidů} / \href{http://www.bioinformatics.org/sms2/random\_protein.html}{aminokyselinami}) a vyzkoušejte si vliv vstupních
parametrů (jako jsou typ skórovací matice nebo velikost posuvného okýnka) na
výsledek dotplot grafu.

\subsection{Region s nízkou složitostí}
\label{sec:org97af557}
Tento příklad ukazuje vliv regionu s nízkou složitostí na výsledek dotplot
grafu.

\begin{enumerate}
\item Jako vstupní sekvenci použijte antigenu \href{data/dotlet\_lowcomp.txt}{Plasmodium falciparum} (parazit
způsobující malárii) a zarovnejte ji vůči sobě samé.
\item Zvolte matici Blosum45 a velikost okýnka 15.
\item Spusťte výpočet a upravte si světlost výsledného grafu např. na -40 a +40
(posuvníky pod histogramem).
\item Ve výsledném grafu by jste měli spatřit tmavou oblast odpovídající sekvenci
identických znaků. Přemístěním kurzoru do tmavé oblasti se vám objeví ve
spodní části okna podrobnější výpis sekvence.
\end{enumerate}

\emph{Otázky:}

\begin{itemize}
\item Jak dlouhá a z jakých znaků se skládá uvedená oblast?
\begin{itemize}
\item 191-225, znaky S
\end{itemize}
\end{itemize}

\subsection{Opakování}
\label{sec:org9a8af80}
Tento příklad ukazuje zajímavé vlastnosti, kterých si můžete všimnout
při zarovnání sekvence oproti sobě samé.

\begin{enumerate}
\item Jako vstupní sekvenci použijte \href{data/dotlet\_rep.txt}{SLIT protein} Drosophila melanogaster (vinná
muška).
\item Zvolte matici Blosum62 a velikost okýnka 15.
\item Spusťte výpočet a upravte si světlost výsledného grafu např. na 0 a +90.
\item Ve výsledném grafu by jste měli spatřit dva shluky, jeden tvořený z delších a
druhý z kratších opakujících se podřetězců (zkráceně opakování).
\end{enumerate}

\emph{Otázky:}
\begin{itemize}
\item Jak v dotplot grafu poznáte opakování?
\begin{itemize}
\item Krátké (symetrické?) diagonální pruhy i mimo hlavní diagonálu
\end{itemize}
\item Kolik jste napočítali delších a kolik kratších opakování?
\begin{itemize}
\item Pokud počítám jen dolní trojúhelníkovou matici, tak 12 regionů s
několikanásobným opakováním (krátké diagonální čáry)
\item Delších opakování bych z diagramu napočítal čtyři, ale tam záleží kritériích
opakování.
\end{itemize}
\end{itemize}

\section{Dynamické programování}
\label{sec:orge6d4ffe}
\subsection{Míra identity u náhodných biologických sekvencí}
\label{sec:orgc6dfdd0}
\begin{enumerate}
\item Vytvořte si 2 náhodné sekvence \href{http://www.bioinformatics.org/sms2/random\_dna.html}{nukleotidů} a \href{http://www.bioinformatics.org/sms2/random\_protein.html}{aminokyselinami}.
\item Dále budeme pracovat s online nástroji pro globální (\href{https://www.ebi.ac.uk/Tools/psa/emboss\_needle/nucleotide.html}{Needle Nucl} a \href{https://www.ebi.ac.uk/Tools/psa/emboss\_needle/}{Needle
Prot}) a lokální (\href{https://www.ebi.ac.uk/Tools/psa/lalign/nucleotide.html}{LALIGN Nucl} a \href{https://www.ebi.ac.uk/Tools/psa/lalign/}{LALIGN Prot} ) zarovnání
nukleotidových/aminokyselinových sekvencí.
\item Pomocí nástrojů Needle resp. LALIGN vykonajte zarovnání vybraných
nukleotidových/aminokyselinových sekvencí. Při zarovnávání aminokyselinových
sekvencí zvolte matici BLOSUM50. Jinak ponechte původní nastavení. (LALIGN
nabízí i pěkný grafický výstup v záložce \emph{Visual Output})
\item Sledujte hodnoty vyjadřující míru identity (Needle) a parametr E (pouze u
LALIGN).
\end{enumerate}

\emph{Otázky:}
\begin{itemize}
\item Co vyjadřuje parametr E?
\begin{itemize}
\item Očekávaný počet shod v databázi o N sekvencích (děleno N). Hodnota blížící
se 0 odpovídá významné shodě, hodnota blížící se 1 znamená buď nevýznamnou
(malou) shodu, nebo shodu krátké sekvence s databázi dlouhých sekvencí, kde
by byla běžná čistě pravděpodobnostně.
\end{itemize}
\item Jaké míry identity a parametru E je obvykle dosahováno u náhodných
nukleotidových sekvencí?
\begin{itemize}
\item Identita 0.3 globálně
\item Identita 0.3, E 1 pro náhodné shody v podsekvencích
\end{itemize}
\item Jaké míry identity a parametru E je obvykle dosahováno u náhodných
sekvencí aminokyselin?
\begin{itemize}
\item Identita 0.15 globálně
\item Identita 0.3, E 1 pro náhodné shody v podsekvencích
\end{itemize}
\end{itemize}

\subsection{Objevení podobnosti mezi onkogeny}
\label{sec:orgba185a2}
Russell Doolittle byl průkopníkem v oblasti algoritmů pro analýzu sekvencí v
druhé polovině 70 a první polovině 80 let. Doolittle používal tehdejší databáze
biologických sekvencí pro své experimenty s geny a hledání jejich funkce na
základě podobnosti. V následujícím cvičení si zopakujeme kroky, které Doolittle
provedl při objevu funkce \textbf{v-mos} onkogenu viru \emph{Moloney Murine Sarcoma}. Ne dlouho
poté, co byl nasekvenován v-mos v Salk Institutu, studovala skupina vědců vztah
mezi \textbf{v-src} onkogen viru \emph{Rous Sarcoma} a v-mos onkogenu. První pokusy o hledání
podobnosti však dopadly neúspěšně.

\begin{enumerate}
\item Pro tenhle úkol budeme potřebovat nástroj \href{https://www.ebi.ac.uk/Tools/st/emboss\_sixpack/}{Sixpack} pro překlad nukleotidové
sekvence na aminokyselinovou.
\item S použitím nástrojů Needle resp. LALIGN pro globální resp. lokální zarovnání
analyzujte podobnost obou nukleotidových sekvencí \href{data/vmos.fasta}{vmos.fasta} a \href{data/src.fasta}{src.fasta}.
\item Na základě míry identity a parametru E zhodnoťte výsledky zarovnání.
\item Dále použijte nástroj Sixpack který dokáže přeložit sekvenci nukleotidů na
aminokyseliny pro překlad sekvencí \emph{v-mos} a \emph{v-src}.  Ponechte původní
nastavení. Všimněte si, že nástroj Sixpack provede celkem 6 překladů s
ohledem na různou počáteční pozici překladu fram-u. Tři překlady provede s
přímou sekvencí a tři překlady provede s komplementární a reverzovanou
sekvencí (simulace komplementárního vlákna DNA).
\item Z šesti překladů obou sekvencí vyberte takové, které by mohli vést na co
největší míru podobnosti (říďte se tabulkou na konci výstupu, výsledné
přeložené sekvence najdete v záložce \emph{Tool Output}).
\item Proveďte globální a lokální zarovnání přeložených sekvencí pomocí nástrojů
Needle/LALIGN.
\item Na základě míry identity a parametru E zhodnoťte výsledky zarovnání.
\item Pokud si nejste jisti, zda jste vybrali správné fram-y v bodě 5, vyberte jiné
a opakujte experiment. Nápověda: hledáme takové posuny, které vedou na
překlad jednoho genu v protein (jeden start/stop kodon)
\end{enumerate}

\emph{Otázky:}
\begin{itemize}
\item Jaké fram-y jste vybrali?
\begin{itemize}
\item Frame 1 v obou případech - jediné souvislé fram-y bez více ORF
\end{itemize}
\item Jak hodnotíte výsledky zarovnání nukleotidových sekvencí?
\begin{itemize}
\item Nepříliš přesvědčivé
\end{itemize}
\item Nalezli jste lepší zarovnání v případě přeložených sekvencí?
\begin{itemize}
\item Lokální zarovnání o délce 582 s \(E(1) = 4.6e-07\)
\item Globální zarovnání se skórem 241
\end{itemize}
\end{itemize}

\section{BLAST}
\label{sec:org1f165fa}
\subsection{Hledání původu DinoDNA}
\label{sec:org273d1c2}
\begin{enumerate}
\item Film Michael Crichtona o klonování dinosaurů, Jurský park, ukazuje domnělou
DNA sekvenci dinosaura. Identifikujte skutečný zdroj \href{data/dino\_dna1.fasta}{této} DNA sekvence s
využitím programu \href{https://blast.ncbi.nlm.nih.gov/Blast.cgi?PROGRAM=blastn\&PAGE\_TYPE=BlastSearch\&LINK\_LOC=blasthome}{BLAST} a NCBI databáze všech nukleotidů \textbf{nr}.
\item Vědec NCBI Mark Boguski však upozornil na to, že jeho sekvence byla určitě
kontaminovaná a zásobil Michaela Crichtona lepší sekvencí, pro pokračování
tohoto filmu z názvem The Lost World. Identifikujte zdroj \href{data/dino\_dna2.fasta}{této} sekvence.
\end{enumerate}

\emph{Otázky:}
\begin{itemize}
\item Nalezl Mark lepší sekvenci než Michael? Proč?
\begin{enumerate}
\item Michael - 99\% se shoduje s "Escherichia coli strain Mach1 plasmid pSS1129, complete
sequence" s \(E(1) = 3e-117\)
\item Mark - 66\% se shoduje s "Gallus gallus GATA binding protein 1" s \(E(1) = 0\) (úplná shoda)
\item Na takovou otázku se špatně odpovídá - ani jedna není dobrá pro klonování
dinosaurů. Michaelova sekvence je DNA bakterie E. coli a tudíž naprosto
irelevantní, Markova sekvence je asi relevantnější, obsahuje jeden
konkrétní gen, který se mohl vyskytovat i v DNA dinosaura, pokud uvažujeme
příbuznost \emph{Gallus gallus}, ale ani jedna z nich není sekvence, která by
pomohla rekonstruovat DNA dinosaura. Musím tedy asi odpovědět, že Markova
sekvence je lepší, GATA1 protein, ač ve verzi Gallus gallus, je v
přítomnosti dalších fragmentů DNA užitečnější.
\end{enumerate}
\item Mark zabudoval do své sekvence také své jméno MARK. Nalezněte toto jméno v
sekvenci.
\begin{itemize}
\item Je součástí "DinoDNA\(_{\text{1}}\)\(_{\text{ORF1}}\)  Translation of DinoDNA in frame 1, ORF 1,
threshold 1, 358aa", polovina třetího řádku:
\end{itemize}
\end{itemize}
\begin{minted}[]{text}
>Translation of DinoDNA in frame 1, ORF 1, threshold 1, 358aa
EFRKRARDKSWHQIQLEIRTDVWQLPQRIHWKCITYPMGAMEFVALGGPDAGSPTPFPDE
AGAFLGLGGGERTEAGGLLASYPPSGRVSLVPWADTGTLGTPQWVPPATQMEPPHYLELL
QPPRGSPPHPSSGPLLPLSSGPPPCEARECVMARKNCGATATPLWRRDGTGHYLCNWASA
CGLYHRLNGQNRPLIRPKKRLLVSKRAGTVCSHERENCQTSTTTLWRRSPMGDPVCNNIH
ACGLYYKLHQVNRPLTMRKDGIQTRNRKVSSKGKKRRPPGGGNPSATAGGGAPMGGGGDP
SMPPPPPPPAAAPPQSDALYALGPVVLSGHFLPFGNSGGFFGGGAGGYTAPPGLSPQI
\end{minted}

\subsection{Hledání komplementárních sekvencí}
\label{sec:orgefe9589}
\begin{enumerate}
\item S využitím databáze \href{http://www.ncbi.nlm.nih.gov/genbank/}{NCBI GenBank} si stáhněte sekvenci nukleotidů libovolného
lidského genu napr. KRAS (postačí prvních 1000 znaků genu)
\item S využitím následujícího webového \href{http://www.bioinformatics.org/sms/rev\_comp.html}{nástroje} si ke vstupnímu genu vytvořte:
\begin{itemize}
\item reverzní sekvenci,
\item komplementární sekvenci,
\item reverzní+komplementární sekvenci.
\end{itemize}

\item S využitím \href{http://blast.ncbi.nlm.nih.gov/Blast.cgi}{BLASTu} vyhledejte v \textbf{nr} databázi všechny výskyty vytvořených
sekvencí
\end{enumerate}

\emph{Otázky:}
\begin{itemize}
\item Shodují se výsledky pro všechny alternativy vstupní sekvence? Zdůvodněte proč.
\begin{itemize}
\item Výsledky BLAST jsou shodné pro původní a reverse-complement DNA (GTPase), stejně tak
pro reverse a complement DNA (OATP-B promotor)
\item DNA čteme v pořadí 5-3 a reverse nebo complement verze jsou obě v 3-5
pořadí, BLAST tedy zřejmě hledá i normální i reverse-complement verze, aspoň
pro blastn.
\end{itemize}
\end{itemize}
\end{document}